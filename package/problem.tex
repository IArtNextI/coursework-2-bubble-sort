\begin{problem}{Bubble sort}{}{}{}

Перед вами поставили $n$ стаканов, изначально в каждом из них может быть либо $m$, либо $0$ разноцветных шариков.

Стаканы достаточно узки, поэтому шарики находятся один над другим - и в любой момент времени достать можно только самый верхний. Иначе говоря, состояние стакана в любой момент времени представимо в виде массива чисел: $a_1, ..., a_t$, где $a_i$ - это цвет $i$-го шарика снизу в стакане.

Ваша задача - сделать так, чтобы шарики одного цвета оказались в одном стакане.

Пусть вы хотите переложить верхний шар из стакана $i$ в стакан $j$. Вы можете это сделать только в случае, если выполнено одно из двух условий:
\begin{itemize}
    \item{Стакан $j$ пустой}
    \item{В стакане $j$ сейчас $\leq m - 1$ шар, и цвет верхнего шара $j$-го стакана совпадает с цветом верхнего шара $i$-го стакана}
\end{itemize}

В первой строке входного файла указано число $t$ - количество наборов входных данных в файле

Описание каждого набора начинается со строки, содержащей числа $n$ и $m$ - количество стаканов и максимльно возможное количество шариков в одном стакане соответственно.

В последующих $n$ строках указано содержание стаканов в формате:
Сначала указано число $c_i$ - количество элементов в текущем стакане, после чего указаны $c_i$ чисел - цвета шариков в стакане, в порядке от самого нижнего до самого верхнего.
Выведите $t$ решений для наборов входных данных в следующем формате:

В первой строке каждого решения набора данных выведите число $k \leq 10^{9}$ - количество действий для сортировки в Вашем решении
В следующих $k$ строках выведите сами действия - по два числа $(x_i, y_i)$ - операция перекладывания верхнего шара из стакана $x_i$ в стакан $y_i$.

Оценка решения вычисляется по следующей формуле:

Введем функцию $f(solution) = \sum_{i=1}^n cnt_i^2$. Здесь $cnt_i$ - количество различных элементов в итоговом стакане.

Введем функцию $g(solution) = \frac{m\sqrt{n - \#empty} - \sqrt{f(solution) - n + \#empty}}{m\sqrt{n - \#empty}}$, где $\#empty$ - количество пустых стаканов. 

Оценкой за решение одного набора входных данных будет величина $10\cdot \left(\frac{g(solution)}{g(jury\_solution)}\right)^3$, $где jury\_solution$ - это лучшее решение среди всех участников и решения жюри.

В первом тесте $t = 3$. Оценка за этот тест: 30 баллов. Баллы начисляются только в случае, если все выведенные ходы во всех тестах можно сделать. Проверка осуществляется в режиме online (результат виден сразу).

Во втором тесте $t = 7$. Оценка за этот тест: 70 баллов. Баллы начисляются только в случае, если все выведенные ходы во всех тестах можно сделать. Во время тура проверяется только возможность сделать ходы, описанные во входном файле. Проверка правильности ответа осуществляется в режиме offline (результат виден после окончания тура).

\Examples
\begin{example}
\exmp{
1
4 3
3 1 2 3
3 2 2 1
3 3 3 1
0
}{%
5
2 4
3 4
1 3
1 2
1 4}%
\end{example}

\end{problem}
